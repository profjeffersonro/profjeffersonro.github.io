% Options for packages loaded elsewhere
\PassOptionsToPackage{unicode}{hyperref}
\PassOptionsToPackage{hyphens}{url}
\documentclass[
]{article}
\usepackage{xcolor}
\usepackage{amsmath,amssymb}
\setcounter{secnumdepth}{5}
\usepackage{iftex}
\ifPDFTeX
  \usepackage[T1]{fontenc}
  \usepackage[utf8]{inputenc}
  \usepackage{textcomp} % provide euro and other symbols
\else % if luatex or xetex
  \usepackage{unicode-math} % this also loads fontspec
  \defaultfontfeatures{Scale=MatchLowercase}
  \defaultfontfeatures[\rmfamily]{Ligatures=TeX,Scale=1}
\fi
\usepackage{lmodern}
\ifPDFTeX\else
  % xetex/luatex font selection
\fi
% Use upquote if available, for straight quotes in verbatim environments
\IfFileExists{upquote.sty}{\usepackage{upquote}}{}
\IfFileExists{microtype.sty}{% use microtype if available
  \usepackage[]{microtype}
  \UseMicrotypeSet[protrusion]{basicmath} % disable protrusion for tt fonts
}{}
\makeatletter
\@ifundefined{KOMAClassName}{% if non-KOMA class
  \IfFileExists{parskip.sty}{%
    \usepackage{parskip}
  }{% else
    \setlength{\parindent}{0pt}
    \setlength{\parskip}{6pt plus 2pt minus 1pt}}
}{% if KOMA class
  \KOMAoptions{parskip=half}}
\makeatother
\ifLuaTeX
\usepackage[bidi=basic]{babel}
\else
\usepackage[bidi=default]{babel}
\fi
\babelprovide[main,import]{brazilian}
% get rid of language-specific shorthands (see #6817):
\let\LanguageShortHands\languageshorthands
\def\languageshorthands#1{}
\setlength{\emergencystretch}{3em} % prevent overfull lines
\providecommand{\tightlist}{%
  \setlength{\itemsep}{0pt}\setlength{\parskip}{0pt}}
\usepackage{bookmark}
\IfFileExists{xurl.sty}{\usepackage{xurl}}{} % add URL line breaks if available
\urlstyle{same}
\hypersetup{
  pdftitle={Introdução à Relatividade},
  pdfauthor={Jefferson Rodrigues de Oliveira},
  pdflang={pt-BR},
  hidelinks,
  pdfcreator={LaTeX via pandoc}}

\title{Introdução à Relatividade}
\author{Jefferson Rodrigues de Oliveira}
\date{2025}

\begin{document}
\maketitle

{
\setcounter{tocdepth}{3}
\tableofcontents
}
\section{Introdução à
Relatividade}\label{introduuxe7uxe3o-uxe0-relatividade}

A Teoria da Relatividade, formulada por Albert Einstein no início do
século XX, revolucionou a física ao modificar conceitos fundamentais
como espaço e tempo. Ela se divide em dois grandes pilares:

\begin{itemize}
\tightlist
\item
  \textbf{Relatividade Restrita (1905)}: Trata da física em referenciais
  inerciais e introduz o conceito de dilatação temporal, contração do
  espaço e equivalência entre massa e energia.
\item
  \textbf{Relatividade Geral (1915)}: Generaliza a teoria para
  referenciais não inerciais e descreve a gravidade como a curvatura do
  espaço-tempo.
\end{itemize}

\subsection{Princípios Fundamentais da Relatividade
Restrita}\label{princuxedpios-fundamentais-da-relatividade-restrita}

\begin{enumerate}
\def\labelenumi{\arabic{enumi}.}
\tightlist
\item
  \textbf{Princípio da Relatividade}: As leis da física são as mesmas em
  todos os referenciais inerciais.
\item
  \textbf{Invariância da Velocidade da Luz}: A velocidade da luz no
  vácuo é constante para todos os observadores, independentemente de
  seus referenciais.
\end{enumerate}

\subsection{Equações Fundamentais}\label{equauxe7uxf5es-fundamentais}

A relação entre energia e massa é dada pela famosa equação:

\[
E = mc^2
\]

Além disso, a dilatação temporal é descrita por:

\[
\Delta t' = \frac{\Delta t}{\sqrt{1 - v^2/c^2}}
\]

onde \(\Delta t'\) é o tempo medido pelo observador em movimento e \(v\)
é a velocidade relativa entre referenciais.

\subsection{Conclusão}\label{conclusuxe3o}

A Teoria da Relatividade trouxe implicações profundas, como a
possibilidade de viagens no tempo (relativas) e a explicação de
fenômenos cósmicos como buracos negros e ondas gravitacionais.

\end{document}
