% Options for packages loaded elsewhere
\PassOptionsToPackage{unicode}{hyperref}
\PassOptionsToPackage{hyphens}{url}
\documentclass[
  a4paper,
  12pt]{article}
\usepackage{xcolor}
\usepackage[top=20mm,left=40mm,bottom=20mm,right=40mm]{geometry}
\usepackage{amsmath,amssymb}
\setcounter{secnumdepth}{5}
\usepackage{iftex}
\ifPDFTeX
  \usepackage[T1]{fontenc}
  \usepackage[utf8]{inputenc}
  \usepackage{textcomp} % provide euro and other symbols
\else % if luatex or xetex
  \usepackage{unicode-math} % this also loads fontspec
  \defaultfontfeatures{Scale=MatchLowercase}
  \defaultfontfeatures[\rmfamily]{Ligatures=TeX,Scale=1}
\fi
\usepackage{lmodern}
\ifPDFTeX\else
  % xetex/luatex font selection
\fi
% Use upquote if available, for straight quotes in verbatim environments
\IfFileExists{upquote.sty}{\usepackage{upquote}}{}
\IfFileExists{microtype.sty}{% use microtype if available
  \usepackage[]{microtype}
  \UseMicrotypeSet[protrusion]{basicmath} % disable protrusion for tt fonts
}{}
\makeatletter
\@ifundefined{KOMAClassName}{% if non-KOMA class
  \IfFileExists{parskip.sty}{%
    \usepackage{parskip}
  }{% else
    \setlength{\parindent}{0pt}
    \setlength{\parskip}{6pt plus 2pt minus 1pt}}
}{% if KOMA class
  \KOMAoptions{parskip=half}}
\makeatother
\ifLuaTeX
\usepackage[bidi=basic]{babel}
\else
\usepackage[bidi=default]{babel}
\fi
\babelprovide[main,import]{brazilian}
% get rid of language-specific shorthands (see #6817):
\let\LanguageShortHands\languageshorthands
\def\languageshorthands#1{}
\setlength{\emergencystretch}{3em} % prevent overfull lines
\providecommand{\tightlist}{%
  \setlength{\itemsep}{0pt}\setlength{\parskip}{0pt}}
\usepackage{cancel}
\usepackage[version=4]{mhchem}
\usepackage{tgtermes}
\usepackage{gensymb}
\DeclareUnicodeCharacter{00B0}{\degree}
\usepackage{caption}
\captionsetup{font=footnotesize}
\usepackage{indentfirst}
\setlength{\parindent}{1cm}
\usepackage{bookmark}
\IfFileExists{xurl.sty}{\usepackage{xurl}}{} % add URL line breaks if available
\urlstyle{same}
\hypersetup{
  pdftitle={O Paradoxo do Gato de Schrödinger},
  pdfauthor={Jefferson Rodrigues de Oliveira},
  pdflang={pt-BR},
  hidelinks,
  pdfcreator={LaTeX via pandoc}}

\title{O Paradoxo do Gato de Schrödinger}
\author{Jefferson Rodrigues de Oliveira}
\date{24/11/2024}

\begin{document}
\maketitle

{
\setcounter{tocdepth}{3}
\tableofcontents
}
\section{O Gato de Schrödinger}\label{o-gato-de-schruxf6dinger}

O experimento mental do \textbf{Gato de Schrödinger} foi proposto por
Erwin Schrödinger em 1935 para ilustrar o problema da interpretação da
mecânica quântica.

\subsection{O Experimento Mental}\label{o-experimento-mental}

Schrödinger imaginou um gato dentro de uma caixa fechada com um
mecanismo quântico:

\begin{itemize}
\tightlist
\item
  Um átomo radioativo pode decair ou não dentro de um intervalo de
  tempo.
\item
  O decaimento ativa um detector que libera veneno, matando o gato.
\item
  Se não houver decaimento, o gato permanece vivo.
\end{itemize}

Segundo a \textbf{interpretação de Copenhague}, antes da observação, o
gato está em um estado de \textbf{superposição}:

\[
|\psi\rangle = \frac{1}{\sqrt{2}} (|vivo\rangle + |morto\rangle)
\]

Somente ao abrir a caixa, a função de onda ``colapsa'' e o gato passa a
estar definitivamente vivo ou morto.

\subsection{Interpretações da Mecânica
Quântica}\label{interpretauxe7uxf5es-da-mecuxe2nica-quuxe2ntica}

\begin{enumerate}
\def\labelenumi{\arabic{enumi}.}
\tightlist
\item
  \textbf{Copenhague}: O sistema fica em superposição até a observação.
\item
  \textbf{Muitos Mundos}: Cada possibilidade gera um universo paralelo
  onde o gato está vivo em um e morto no outro.
\item
  \textbf{De Broglie-Bohm}: A partícula sempre tem um estado definido,
  mas segue uma trajetória oculta.
\end{enumerate}

\subsection{Conclusão}\label{conclusuxe3o}

O experimento de Schrödinger é um símbolo das interpretações da mecânica
quântica e dos paradoxos associados à medição e à superposição.

\end{document}
