% Options for packages loaded elsewhere
\PassOptionsToPackage{unicode}{hyperref}
\PassOptionsToPackage{hyphens}{url}
\documentclass[
  a4paper,
  12pt]{article}
\usepackage{xcolor}
\usepackage[top=20mm,left=40mm,bottom=20mm,right=40mm]{geometry}
\usepackage{amsmath,amssymb}
\setcounter{secnumdepth}{5}
\usepackage{iftex}
\ifPDFTeX
  \usepackage[T1]{fontenc}
  \usepackage[utf8]{inputenc}
  \usepackage{textcomp} % provide euro and other symbols
\else % if luatex or xetex
  \usepackage{unicode-math} % this also loads fontspec
  \defaultfontfeatures{Scale=MatchLowercase}
  \defaultfontfeatures[\rmfamily]{Ligatures=TeX,Scale=1}
\fi
\usepackage{lmodern}
\ifPDFTeX\else
  % xetex/luatex font selection
\fi
% Use upquote if available, for straight quotes in verbatim environments
\IfFileExists{upquote.sty}{\usepackage{upquote}}{}
\IfFileExists{microtype.sty}{% use microtype if available
  \usepackage[]{microtype}
  \UseMicrotypeSet[protrusion]{basicmath} % disable protrusion for tt fonts
}{}
\makeatletter
\@ifundefined{KOMAClassName}{% if non-KOMA class
  \IfFileExists{parskip.sty}{%
    \usepackage{parskip}
  }{% else
    \setlength{\parindent}{0pt}
    \setlength{\parskip}{6pt plus 2pt minus 1pt}}
}{% if KOMA class
  \KOMAoptions{parskip=half}}
\makeatother
\ifLuaTeX
\usepackage[bidi=basic]{babel}
\else
\usepackage[bidi=default]{babel}
\fi
\babelprovide[main,import]{brazilian}
% get rid of language-specific shorthands (see #6817):
\let\LanguageShortHands\languageshorthands
\def\languageshorthands#1{}
\setlength{\emergencystretch}{3em} % prevent overfull lines
\providecommand{\tightlist}{%
  \setlength{\itemsep}{0pt}\setlength{\parskip}{0pt}}
\usepackage{cancel}
\usepackage[version=4]{mhchem}
\usepackage{tgtermes}
\usepackage{gensymb}
\DeclareUnicodeCharacter{00B0}{\degree}
\usepackage{caption}
\captionsetup{font=footnotesize}
\usepackage{indentfirst}
\setlength{\parindent}{1cm}
\usepackage{bookmark}
\IfFileExists{xurl.sty}{\usepackage{xurl}}{} % add URL line breaks if available
\urlstyle{same}
\hypersetup{
  pdftitle={Algoritmos Quânticos},
  pdfauthor={Jefferson Rodrigues de Oliveira},
  pdflang={pt-BR},
  hidelinks,
  pdfcreator={LaTeX via pandoc}}

\title{Algoritmos Quânticos}
\author{Jefferson Rodrigues de Oliveira}
\date{01/02/2025}

\begin{document}
\maketitle

{
\setcounter{tocdepth}{3}
\tableofcontents
}
\section{Algoritmos Quânticos}\label{algoritmos-quuxe2nticos}

Os \textbf{algoritmos quânticos} exploram as propriedades da computação
quântica, como superposição e emaranhamento, para resolver problemas de
forma mais eficiente do que algoritmos clássicos.

\subsection{Fundamentos da Computação
Quântica}\label{fundamentos-da-computauxe7uxe3o-quuxe2ntica}

\begin{itemize}
\tightlist
\item
  \textbf{Qubits}: Diferentemente dos bits clássicos (0 ou 1), os qubits
  podem estar em \textbf{superposição} de ambos os estados.
\item
  \textbf{Emaranhamento}: Qubits podem ser correlacionados de forma que
  o estado de um afeta o do outro instantaneamente.
\item
  \textbf{Portas Lógicas Quânticas}: Operam sobre qubits de maneira
  diferente das portas lógicas clássicas.
\end{itemize}

\subsection{Algoritmos Principais}\label{algoritmos-principais}

\begin{enumerate}
\def\labelenumi{\arabic{enumi}.}
\tightlist
\item
  \textbf{Algoritmo de Deutsch-Jozsa}: Resolve um problema específico em
  tempo constante, enquanto um algoritmo clássico necessitaria de
  múltiplas consultas.
\item
  \textbf{Algoritmo de Shor}: Fatoriza números inteiros em tempo
  polinomial, ameaçando a segurança da criptografia RSA.
\item
  \textbf{Algoritmo de Grover}: Encontra um elemento em uma lista não
  ordenada em tempo \(O(\sqrt{N})\), mais rápido que a busca clássica
  \(O(N)\).
\end{enumerate}

\subsection{Aplicações Futuras}\label{aplicauxe7uxf5es-futuras}

Os algoritmos quânticos prometem avanços em áreas como:

\begin{itemize}
\tightlist
\item
  \textbf{Criptografia e segurança cibernética}
\item
  \textbf{Simulação de moléculas para química e medicina}
\item
  \textbf{Otimização de sistemas complexos}
\end{itemize}

\subsection{Conclusão}\label{conclusuxe3o}

Embora ainda em desenvolvimento, a computação quântica tem o potencial
de revolucionar a ciência e a tecnologia, resolvendo problemas
intratáveis para computadores clássicos.

\end{document}
